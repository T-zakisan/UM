
\documentclass[11pt]{ltjsarticle}
\usepackage{tcolorbox}
\usepackage{fancyhdr}
\usepackage{lastpage}
\usepackage{xcolor} % 色指定用


\usepackage{amsmath}
\usepackage{amssymb}
\usepackage{amsfonts}
\usepackage{latexsym}


% レイアウト設定
% \setlength{ \hoffset }          { 0truemm }
% \setlength{ \voffset }          { -15.4truemm }
% \setlength{ \oddsidemargin }    { 0truemm }
% \setlength{ \topmargin }        { 13truemm }
% \setlength{ \headheight }       { 10truemm }
% \setlength{ \headsep }          { 5truemm }
% \setlength{ \textwidth }        { 100truemm }
% \setlength{ \marginparsep }     { 0truemm }
% \setlength{ \marginparwidth }   { 0truemm }
% \setlength{ \footskip }         { 17.5truemm }



% デフォルトの余白設定
\usepackage[top     =20truemm,
            bottom  =18truemm,
            left    =23truemm, 
            right   =23truemm]{geometry}
\setlength{ \topmargin }    { -20mm }
\setlength{ \headheight }   { 23mm }
\setlength{ \headsep }      { -8mm }


% 和文をゴシック体に変更 ※通常オフ
%\renewcommand{\kanjifamilydefault}{\gtdefault} % 和文フォントをゴシック体に



% 1ページあたりの行数
\newcommand{\linesparpage}[1]{%
    \setlength{\baselineskip}{\dimexpr \textheight / #1 \relax}%
}


% ページ番号の形式設定
\pagestyle{fancy}
\fancyhf{} % 全てのヘッダー・フッターをクリア
\fancyfoot[C]{\bfseries\ttfamily\ifnum\value{section}=0 S-\thepage\else \thesection-\thepage \fi} % ページ番号
\renewcommand{\headrulewidth}{0pt} % デフォルトのヘッダー罫線を非表示
\renewcommand{\footrulewidth}{0.4pt} % フッターに罫線を表示


% ヘッダー部にセクション名を表示
\fancyhead[C]{%
    \begin{tcolorbox}[colback=gray!100, colframe=black, sharp corners, 
                        boxrule=0pt, toprule=0.5pt, bottomrule=0.5pt, 
                        width=\linewidth, boxsep=0pt ]
        \centering\bfseries\ttfamily\textcolor{white}{\leftmark}
    \end{tcolorbox}
}


% sectionの再定義
\makeatletter
\renewcommand{\section}{%
    \@ifstar{\customsectionstar}{\customsection}} % *付きと無しで分岐

% 通常のセクションのカスタマイズ
\newcommand{\customsection}[1]{%
    \cleardoublepage % 右ページから開始
    \newgeometry{top=80truemm, bottom=18truemm, left=23truemm, right=23truemm}
    \thispagestyle{empty} % このページのページ番号を非表示
    \setcounter{page}{0} % ページ番号リセット
    \stepcounter{section} % セクション番号をインクリメント
    \leftskip = 0pt%
    \markboth{\protect\textcolor{white}{\thesection 章 #1}}{} % セクション名をヘッダーに設定
    \begin{tcolorbox}[colback=gray!20, colframe=black, sharp corners, 
                        top=5mm, left=0mm, bottom=5mm, right=0mm,
                        boxrule=0pt, toprule=1.5pt, bottomrule=1.5pt, 
                        width=\linewidth ] % 上下に1ptの罫線
        \fontsize{34pt}{34pt}\selectfont\bfseries\gtfamily\centering \thesection 章 \hspace{1ex} #1
    \end{tcolorbox}%
    \clearpage
    \restoregeometry
}

% *付きのセクションのカスタマイズ
\newcommand{\customsectionstar}[1]{%
    \cleardoublepage % 右ページから開始
    \newgeometry{top=80truemm, bottom=18truemm, left=23truemm, right=23truemm}
    \thispagestyle{empty} % このページのページ番号を非表示
    \setcounter{page}{0} % ページ番号リセット
    \leftskip = 0pt%
    \markboth{\protect\textcolor{white}{#1}}{} % セクション名をヘッダーに設定
    \begin{tcolorbox}[colback=gray!20, colframe=black, sharp corners, 
                        top=5mm, left=0mm, bottom=5mm, right=0mm,
                        boxrule=0pt, toprule=1.5pt, bottomrule=1.5pt, 
                        width=\linewidth ]
        \fontsize{34pt}{34pt}\selectfont\bfseries\gtfamily\centering #1
    \end{tcolorbox}%
    \clearpage
    \restoregeometry
}

% subsectionの再定義:(1) 形式
\renewcommand{\subsection}[1]{%
    \leftskip = 0pt%
    \refstepcounter{subsection}%
    \noindent{\normalsize\bfseries (\arabic{subsection}) #1}%
    \par\nobreak\vspace{0.5ex}%
}


% subsubsectionの再定義
\renewcommand{\subsubsection}{%
    \@ifstar{\customsubsubsectionstar}{\customsubsubsection}} % *付きと無しで分岐
% a. 形式
\newcommand{\customsubsubsection}[1]{%
    \leftskip = 0pt%
    \refstepcounter{subsubsection}%
    \noindent{\normalsize\bfseries \alph{subsubsection}. #1}%
    \par\nobreak\vspace{0.5ex}%
    \leftskip = 2em%
}
% ■形式:*付
\newcommand{\customsubsubsectionstar}[1]{%
    \leftskip = 0pt%
    \noindent{\normalsize\bfseries $\blacksquare$ \hspace{0em} #1}%
    \par\nobreak\vspace{0.5ex}%
    \leftskip = 1.5em%
}
\makeatother


% ◯数:数字付きリスト用
\renewcommand{\labelenumi}{\textcircled{\scriptsize \theenumi}}


% enumerateの前後のスペースを調整
\usepackage{enumitem}
\setlist[enumerate]{
    topsep      =0pt,   % 縦方向の余白(前後)
    leftmargin  =2.5em,  % 左のインデント幅(デフォルトは2.5em程度)
    labelindent =0em  % ラベルの位置を調整(必要に応じて)
}


% ◯段組み
\usepackage{multicol}


% レイアウト確認
% \usepackage{layout}


% フォント
% \usepackage{luatexja-fontspec}
% \usepackage{fontspec} % LuaLaTeXを使用している場合
% \setmainfont{"BIZUDGothic"} % フォントを統一


% \usepackage{titlesec} % タイトル書式変更のため

% \partの書式設定
% \titleformat{\part}[block]
%   {\normalfont\huge\bfseries\centering}  % フォントと中央寄せ
%   {}{1em}{}  % パート番号なし、中央寄せに設定

% \titlecontents{part}[0pt] % 目次設定
%   {\normalfont\huge\bfseries\centering}{\contentslabel{2em}}{}{\titlerule*[.5pc]{.}\contentspage}



\begin{document}



%
% 1_cover.tex 表紙
%

\newgeometry{top=82truemm, bottom=18truemm, left=42truemm, right=42truemm}
\thispagestyle{empty} % このページのページ番号を非表示

\centering
\framebox[125.0mm]{
    \begin{minipage}[c][69.0mm][c]{124.0mm} %Boxのサイズ(タテ×ヨコ)
        \vspace*{10mm} %枠上端から1文字目上端までの距離
        \makebox[124.0mm][c]{\Huge{\textbf{\textsf{自動カーバ}}}}   \\  [5.5mm]  %タイトル
        \makebox[124.0mm][c]{\Huge{\textbf{\textsf{AC-R8型}}}}     \\  [12.0mm]    %機種
        \makebox[124.0mm][c]{\Huge{\textbf{\textsf{取扱説明書}}}}   \\  [3.0mm] 
        \makebox[124.0mm][c]{\Huge{\textbf{\textsf{パーツリスト}}}} \\  [5.0mm] 
    \end{minipage}
}

\vspace{3.0truemm}
\textbf{適応号機}\\
\textbf{\texttt{xxxx号機以降}} %適応号機


\centering
\vspace{30.0truemm}
\begin{minipage}[c][25.0mm][c]{110.0mm} %Boxのサイズ(タテ×ヨコ)
    \renewcommand{\labelitemi}{$\ast$}
    \begin{itemize}
        \item ご使用前に必ずお読みください.
        \item 本書は大切に保管し,必要なときにすぐに見られる場所に保管してください.
        \item 将来の参照用として保存してください.
    \end{itemize}
\end{minipage}


\centering
\vspace{17.0truemm}
\textbf{\texttt{発行日:2024年 〇月〇〇日}} \\ [0.0mm] %発行日
\textbf{\texttt{第0.0版}}                  \\ [0.0mm] %版
\textbf{\texttt{Book No. AAAAAAAAAAAAAA}}  \\ [10.0mm] %BookNo
\huge{\textbf{\texttt{範多機械株式会社}}} %会社名
\thispagestyle{empty} %ページ番号非表示
\cleardoublepage %右ページから開始
\restoregeometry %余白設定を戻す
               % 表紙
%
% 2_greetings.tex ごあいさつ
%
\clearpage%

\cleardoublepage %右ページから開始

\setlength{\parindent}{1em}  % 必要ならば字下げを元に戻す
\normalsize % フォントサイズをデフォルトに戻す
\restoregeometry %余白設定を戻す


\part*{ごあいさつ}
\addcontentsline{toc}{part}{ごあいさつ} 

この度は,自動カーバをお買い上げいただきまして,まことにありがとうございます.

本気は,厳しい検索を行って出荷しておりますが,取り扱いを誤ったり,日頃の点検・整備などを怠りますと,優れた機械でも故障を起こし,時には人身事故や重大な破損事故を招くことになります.

本書は,安全で正しい運転操作や応急処置および点検・整備などの必要な事柄を説明しておりますので,必ずお読みください.
また,取り扱いを十分にご存知の方も本機独自の機構や取り扱いがございますので,本機を使用する前に本書を熟読し,「安全運転・正しい管理」をしてください.

本書は,機械の一部として供給され,機械を取り扱う上で,重要なものです.
本書を紛失,または破損し読めなくなった場合は,直ちに新しい取扱説明書と交換してください.

エンジン・付属品に関することは,別冊の取扱説明書をご参照ください.




\part*{整備とサービスについて}
\addcontentsline{toc}{part}{整備とサービスについて}

ご使用中の故障やその他,ご不明な点およびサービスに関するご用命は,弊社またはお買い上げいただきました販売店にお気軽にご相談ください.
その際に,製品名・製品型式・機械番号・エンジン名称・エンジン番号を併せてご連絡ください.

保証に関することは,「保証証券」に記載しておりますので,ご使用の前に必ずお読みください.
本書は,各取扱説明書と共に大切に保管してください.


% 以降,字下げを行わない
\setlength{\parindent}{0pt}
           % ごあいさつ,整備とサービスについて
%
% 3_top.tex もくじ
%


\newpage
\null %なにもない
\thispagestyle{empty} % このページのページ番号を非表示

\cleardoublepage  % 改ページ:新しいページで目次
% \sectionB{目次}
% \tableofcontents
% \tocSPstyle


\sectionB{目次}
\tableofcontents
\tocSPstyle


                 % もくじ
%
% 4_safety-precautions.tex : 安全上の注意事項
%

\sectionC{安全上の注意事項}

\subsection{はじめに}

本書は,機会を安全に正しくご使用いただくためのものです.
\vspace*{-1em}
\begin{itemize}
  \item 一つ目
  \item 二つ目
  \item 三つ目
\end{itemize}


  % 安全上の注意事項


\end{document}
