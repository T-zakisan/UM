\documentclass[11pt]{ltjsarticle}
\usepackage{tcolorbox}
\usepackage{fancyhdr}
\usepackage{lastpage}
\usepackage{xcolor} % 色指定用


\usepackage{amsmath}
\usepackage{amssymb}
\usepackage{amsfonts}
\usepackage{latexsym}


% レイアウト設定
% \setlength{ \hoffset }          { 0truemm }
% \setlength{ \voffset }          { -15.4truemm }
% \setlength{ \oddsidemargin }    { 0truemm }
% \setlength{ \topmargin }        { 13truemm }
% \setlength{ \headheight }       { 10truemm }
% \setlength{ \headsep }          { 5truemm }
% \setlength{ \textwidth }        { 100truemm }
% \setlength{ \marginparsep }     { 0truemm }
% \setlength{ \marginparwidth }   { 0truemm }
% \setlength{ \footskip }         { 17.5truemm }



% デフォルトの余白設定
\usepackage[top     =20truemm,
            bottom  =18truemm,
            left    =23truemm, 
            right   =23truemm]{geometry}
\setlength{ \headheight }   { 10mm }
\setlength{ \headsep }      { 5mm }

% 和文をゴシック体に変更 ※通常オフ
%\renewcommand{\kanjifamilydefault}{\gtdefault} % 和文フォントをゴシック体に


%字下げを行わない
\setlength{\parindent}{0pt}


% 1ページあたりの行数
\newcommand{\linesparpage}[1]{%
    \setlength{\baselineskip}{\dimexpr \textheight / #1 \relax}%
}


% ページ番号の形式設定
\pagestyle{fancy}
\fancyhf{} % 全てのヘッダー・フッターをクリア
\fancyfoot[C]{\bfseries\ttfamily\ifnum\value{section}=0 S-\thepage\else \thesection-\thepage \fi} % ページ番号
\renewcommand{\headrulewidth}{0pt} % デフォルトのヘッダー罫線を非表示
\renewcommand{\footrulewidth}{0.4pt} % フッターに罫線を表示


% ヘッダー部にセクション名を表示
\fancyhead[C]{%
    \begin{tcolorbox}[colback=gray!100, colframe=black, sharp corners, 
                        boxrule=0pt, toprule=0.5pt, bottomrule=0.5pt, 
                        width=\linewidth, boxsep=0pt ]
        \centering\bfseries\ttfamily\textcolor{white}{\leftmark}
    \end{tcolorbox}
}


% sectionの再定義
\makeatletter
\renewcommand{\section}{%
    \@ifstar{\customsectionstar}{\customsection}} % *付きと無しで分岐

% 通常のセクションのカスタマイズ
\newcommand{\customsection}[1]{%
    \cleardoublepage % 右ページから開始
    \newgeometry{top=80truemm, bottom=18truemm, left=23truemm, right=23truemm}
    \thispagestyle{empty} % このページのページ番号を非表示
    \setcounter{page}{0} % ページ番号リセット
    \stepcounter{section} % セクション番号をインクリメント
    \leftskip = 0pt%
    \markboth{\protect\textcolor{white}{\thesection 章 #1}}{} % セクション名をヘッダーに設定
    \begin{tcolorbox}[colback=gray!20, colframe=black, sharp corners, 
                        top=5mm, left=0mm, bottom=5mm, right=0mm,
                        boxrule=0pt, toprule=1.5pt, bottomrule=1.5pt, 
                        width=\linewidth ] % 上下に1ptの罫線
        \fontsize{34pt}{34pt}\selectfont\bfseries\gtfamily\centering \thesection 章 \hspace{1ex} #1
    \end{tcolorbox}%
    \clearpage
    \restoregeometry
}

% *付きのセクションのカスタマイズ
\newcommand{\customsectionstar}[1]{%
    \cleardoublepage % 右ページから開始
    \newgeometry{top=80truemm, bottom=18truemm, left=23truemm, right=23truemm}
    \thispagestyle{empty} % このページのページ番号を非表示
    \setcounter{page}{0} % ページ番号リセット
    \leftskip = 0pt%
    \markboth{\protect\textcolor{white}{#1}}{} % セクション名をヘッダーに設定
    \begin{tcolorbox}[colback=gray!20, colframe=black, sharp corners, 
                        top=5mm, left=0mm, bottom=5mm, right=0mm,
                        boxrule=0pt, toprule=1.5pt, bottomrule=1.5pt, 
                        width=\linewidth ]
        \fontsize{34pt}{34pt}\selectfont\bfseries\gtfamily\centering #1
    \end{tcolorbox}%
    \clearpage
    \restoregeometry
}

% subsectionの再定義:(1) 形式
\renewcommand{\subsection}[1]{%
    \leftskip = 0pt%
    \refstepcounter{subsection}%
    \noindent{\normalsize\bfseries (\arabic{subsection}) #1}%
    \par\nobreak\vspace{0.5ex}%
}


% subsubsectionの再定義
\renewcommand{\subsubsection}{%
    \@ifstar{\customsubsubsectionstar}{\customsubsubsection}} % *付きと無しで分岐
% a. 形式
\newcommand{\customsubsubsection}[1]{%
    \leftskip = 0pt%
    \refstepcounter{subsubsection}%
    \noindent{\normalsize\bfseries \alph{subsubsection}. #1}%
    \par\nobreak\vspace{0.5ex}%
    \leftskip = 2em%
}
% ■形式:*付
\newcommand{\customsubsubsectionstar}[1]{%
    \leftskip = 0pt%
    \noindent{\normalsize\bfseries $\blacksquare$ \hspace{0em} #1}%
    \par\nobreak\vspace{0.5ex}%
    \leftskip = 1.5em%
}
\makeatother


% ◯数:数字付きリスト用
\renewcommand{\labelenumi}{\textcircled{\scriptsize \theenumi}}


% enumerateの前後のスペースを調整
\usepackage{enumitem}
\setlist[enumerate]{
    topsep      =0pt,      % 縦方向の余白(前後)
    leftmargin  =2.5em,  % 左のインデント幅(デフォルトは2.5em程度)
    labelindent =0em  % ラベルの位置を調整(必要に応じて)
}


% ◯段組み
\usepackage{multicol}


% レイアウト確認
\usepackage{layout}



\begin{document}
\linesparpage{42}
\layout


\section*{安全上の注意事項}

\subsubsection{点検・整備時の注意}

\subsubsection*{始業点検の励行}
安全で公開的にお使いいただくために,点検要領にそって始業点検を・・・
安全で公開的にお使いいただくために,点検要領にそって始業点検を・・・
安全で公開的にお使いいただくために,点検要領にそって始業点検を・・・
安全で公開的にお使いいただくために,点検要領にそって始業点検を・・・
安全で公開的にお使いいただくために,点検要領にそって始業点検を・・・
安全で公開的にお使いいただくために,点検要領にそって始業点検を・・・
安全で公開的にお使いいただくために,点検要領にそって始業点検を・・・
安全で公開的にお使いいただくために,点検要領にそって始業点検を・・・
安全で公開的にお使いいただくために,点検要領にそって始業点検を・・・
安全で公開的にお使いいただくために,点検要領にそって始業点検を・・・
安全で公開的にお使いいただくために,点検要領にそって始業点検を・・・
安全で公開的にお使いいただくために,点検要領にそって始業点検を・・・
安全で公開的にお使いいただくために,点検要領にそって始業点検を・・・
安全で公開的にお使いいただくために,点検要領にそって始業点検を・・・
安全で公開的にお使いいただくために,点検要領にそって始業点検を・・・
安全で公開的にお使いいただくために,点検要領にそって始業点検を・・・
安全で公開的にお使いいただくために,点検要領にそって始業点検を・・・
安全で公開的にお使いいただくために,点検要領にそって始業点検を・・・
安全で公開的にお使いいただくために,点検要領にそって始業点検を・・・
安全で公開的にお使いいただくために,点検要領にそって始業点検を・・・
安全で公開的にお使いいただくために,点検要領にそって始業点検を・・・
安全で公開的にお使いいただくために,点検要領にそって始業点検を・・・
安全で公開的にお使いいただくために,点検要領にそって始業点検を・・・
安全で公開的にお使いいただくために,点検要領にそって始業点検を・・・
安全で公開的にお使いいただくために,点検要領にそって始業点検を・・・
安全で公開的にお使いいただくために,点検要領にそって始業点検を・・・
安全で公開的にお使いいただくために,点検要領にそって始業点検を・・・
安全で公開的にお使いいただくために,点検要領にそって始業点検を・・・
安全で公開的にお使いいただくために,点検要領にそって始業点検を・・・
安全で公開的にお使いいただくために,点検要領にそって始業点検を・・・
安全で公開的にお使いいただくために,点検要領にそって始業点検を・・・
安全で公開的にお使いいただくために,点検要領にそって始業点検を・・・
安全で公開的にお使いいただくために,点検要領にそって始業点検を・・・
安全で公開的にお使いいただくために,点検要領にそって始業点検を・・・
安全で公開的にお使いいただくために,点検要領にそって始業点検を・・・
安全で公開的にお使いいただくために,点検要領にそって始業点検を・・・
安全で公開的にお使いいただくために,点検要領にそって始業点検を・・・
安全で公開的にお使いいただくために,点検要領にそって始業点検を・・・
安全で公開的にお使いいただくために,点検要領にそって始業点検を・・・
安全で公開的にお使いいただくために,点検要領にそって始業点検を・・・
安全で公開的にお使いいただくために,点検要領にそって始業点検を・・・
安全で公開的にお使いいただくために,点検要領にそって始業点検を・・・
安全で公開的にお使いいただくために,点検要領にそって始業点検を・・・
安全で公開的にお使いいただくために,点検要領にそって始業点検を・・・
安全で公開的にお使いいただくために,点検要領にそって始業点検を・・・
安全で公開的にお使いいただくために,点検要領にそって始業点検を・・・
安全で公開的にお使いいただくために,点検要領にそって始業点検を・・・
安全で公開的にお使いいただくために,点検要領にそって始業点検を・・・
安全で公開的にお使いいただくために,点検要領にそって始業点検を・・・
安全で公開的にお使いいただくために,点検要領にそって始業点検を・・・
安全で公開的にお使いいただくために,点検要領にそって始業点検を・・・
安全で公開的にお使いいただくために,点検要領にそって始業点検を・・・
安全で公開的にお使いいただくために,点検要領にそって始業点検を・・・
安全で公開的にお使いいただくために,点検要領にそって始業点検を・・・
安全で公開的にお使いいただくために,点検要領にそって始業点検を・・・
安全で公開的にお使いいただくために,点検要領にそって始業点検を・・・
安全で公開的にお使いいただくために,点検要領にそって始業点検を・・・
安全で公開的にお使いいただくために,点検要領にそって始業点検を・・・
安全で公開的にお使いいただくために,点検要領にそって始業点検を・・・
安全で公開的にお使いいただくために,点検要領にそって始業点検を・・・
安全で公開的にお使いいただくために,点検要領にそって始業点検を・・・
安全で公開的にお使いいただくために,点検要領にそって始業点検を・・・
安全で公開的にお使いいただくために,点検要領にそって始業点検を・・・
安全で公開的にお使いいただくために,点検要領にそって始業点検を・・・
安全で公開的にお使いいただくために,点検要領にそって始業点検を・・・
安全で公開的にお使いいただくために,点検要領にそって始業点検を・・・
安全で公開的にお使いいただくために,点検要領にそって始業点検を・・・
安全で公開的にお使いいただくために,点検要領にそって始業点検を・・・
安全で公開的にお使いいただくために,点検要領にそって始業点検を・・・
安全で公開的にお使いいただくために,点検要領にそって始業点検を・・・
安全で公開的にお使いいただくために,点検要領にそって始業点検を・・・
安全で公開的にお使いいただくために,点検要領にそって始業点検を・・・
安全で公開的にお使いいただくために,点検要領にそって始業点検を・・・
安全で公開的にお使いいただくために,点検要領にそって始業点検を・・・
安全で公開的にお使いいただくために,点検要領にそって始業点検を・・・
安全で公開的にお使いいただくために,点検要領にそって始業点検を・・・
安全で公開的にお使いいただくために,点検要領にそって始業点検を・・・
安全で公開的にお使いいただくために,点検要領にそって始業点検を・・・
安全で公開的にお使いいただくために,点検要領にそって始業点検を・・・
安全で公開的にお使いいただくために,点検要領にそって始業点検を・・・
安全で公開的にお使いいただくために,点検要領にそって始業点検を・・・
安全で公開的にお使いいただくために,点検要領にそって始業点検を・・・
安全で公開的にお使いいただくために,点検要領にそって始業点検を・・・
安全で公開的にお使いいただくために,点検要領にそって始業点検を・・・
安全で公開的にお使いいただくために,点検要領にそって始業点検を・・・
安全で公開的にお使いいただくために,点検要領にそって始業点検を・・・
安全で公開的にお使いいただくために,点検要領にそって始業点検を・・・
安全で公開的にお使いいただくために,点検要領にそって始業点検を・・・
安全で公開的にお使いいただくために,点検要領にそって始業点検を・・・
安全で公開的にお使いいただくために,点検要領にそって始業点検を・・・
安全で公開的にお使いいただくために,点検要領にそって始業点検を・・・
安全で公開的にお使いいただくために,点検要領にそって始業点検を・・・
安全で公開的にお使いいただくために,点検要領にそって始業点検を・・・
安全で公開的にお使いいただくために,点検要領にそって始業点検を・・・
安全で公開的にお使いいただくために,点検要領にそって始業点検を・・・
安全で公開的にお使いいただくために,点検要領にそって始業点検を・・・
安全で公開的にお使いいただくために,点検要領にそって始業点検を・・・
安全で公開的にお使いいただくために,点検要領にそって始業点検を・・・
安全で公開的にお使いいただくために,点検要領にそって始業点検を・・・
安全で公開的にお使いいただくために,点検要領にそって始業点検を・・・


\subsubsection*{始業点検の励行}
\begin{enumerate}
    \item コメント1
    \item コメント2
    \item コメント3
    \item コメント1
    \item コメント2
    \item コメント3
    \item コメント1
    \item コメント2
    \item コメント3
    \item コメント1
    \item コメント2
    \item コメント3
    \item コメント1
    \item コメント2
    \item コメント3
    \item コメント1
    \item コメント2
    \item コメント3
    \item コメント1
    \item コメント2
    \item コメント3
    \item コメント1
    \item コメント2
\end{enumerate}



\newpage
\subsubsection*{始業点検の励行}
\begin{multicols}{2}%
\begin{enumerate}
    \item コメント1
    \item コメント2
    \item コメント3
    \item コメント1
    \item コメント2
    \item コメント3
    \item コメント1
    \item コメント2
    \item コメント3
    \item コメント1
    \item コメント2
    \item コメント3
    \item コメント1
    \item コメント2
    \item コメント3
    \item コメント1
    \item コメント2
    \item コメント3
    \item コメント1
    \item コメント2
    \item コメント3
    \item コメント1
    \item コメント2
\end{enumerate}
\end{multicols}

\subsubsection*{始業点検の励行}
\subsubsection*{始業点検の励行}





\newpage
ここに特別なセクションの内容が入ります。

\section{通常のセクション}
ここに通常セクションの内容が入ります。

\subsection{サブセクション1}
ここにサブセクション1の内容が入ります。

\subsubsection{サブサブセクション1a}
ここにサブサブセクション1aの内容が入ります。

\subsubsection{サブサブセクション1b}
ここにサブサブセクション1bの内容が入ります。

\subsection{サブセクション2}
ここにサブセクション2の内容が入ります。

\newpage
\section{通常のセクション2}
ここに通常セクション2の内容が入ります。

\subsection{サブセクション3}
ここにサブセクション3の内容が入ります。

\subsubsection{サブサブセクション3a}
ここにサブサブセクション3aの内容が入ります。

\end{document}
